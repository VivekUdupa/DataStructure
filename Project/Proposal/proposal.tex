\documentclass[12pt]{article}
\usepackage[left=2cm, right=2cm, top=2cm]{geometry}
\usepackage[utf8]{inputenc} 
\usepackage{mdframed} %For framing the title
\usepackage{graphicx} % to include images
\usepackage{amsmath} % For math mode
\usepackage{caption} % For captions
\usepackage{subcaption} % To use caption while using mini page
\usepackage{amssymb} % To use math symbols
\usepackage{multirow} %To combine multiple rows in a table
\usepackage[table]{xcolor} %To color rows / columns in table
\usepackage{titling} %To vertically center the title page
\usepackage{hyperref} %for URL

\title{MATH 8650 \\ Advanced Data Structures \\ Fall 2018\\ \quad \\
	Term Project Proposal \\ Optimization of Bellman Ford Algorithm}
\author{Submitted by: 
\\ Vivek Koodli Udupa 
\\ Sai Harsha Nagabothu}
\date{November - 10, 2018 }

%%To make the title page center vertically centered
%\renewcommand\maketitlehooka{\null\mbox{}\vfill}
%\renewcommand\maketitlehookd{\vfill\null}

\begin{document}
\begin{mdframed}
%Displaying Title
%\begin{titlepage}
\maketitle
%\pagenumbering{gobble}% Remove page numbers (and reset to 1)
%\end{titlepage}
\end{mdframed}
\pagenumbering{arabic}% Arabic page numbers (and reset to 1)

\section{Introduction}
Pathfinding is the plotting of a computer application of the shortest route between two points. It is an essential part of many applications such as video games, robot navigation, road maps etc. Dijkstra's and Bellman Ford are two of many algorithms that are used to find the shortest path. Even though Dijkstra's is faster, Bellman Ford is considered when negative cycles are present in the graph.  
Standard Bellman Ford has a complexity of O(VxE) where V is the vertices and E is the edges of the graph. In this project we will try to improve the runtime of Bellman Ford algorithm.  
\section{Goals}
\begin{enumerate}
	\item Implement Bellman Ford Algorithm
	\item Implement optimized Bellman Ford Algorithm
	\item Design test cases to validate the implementation
	\item Compare the two for performance, accuracy and efficiency
\end{enumerate}

\section{Deliverables}
\begin{enumerate}
	\item Python implementation source code ( Jupyter Notebook )
	\item Report
	\item Project Presentation
\end{enumerate}

\section{References}
[1] \url{https://en.wikipedia.org/wiki/Pathfinding} \\
\\ \noindent
[2] \url{https://en.wikipedia.org/wiki/Bellman\%E2\%80\%93Ford_algorithm} \\
\\ \noindent
[3] Wei Zhang , Hao Chen , Chong Jiang , Lin Zhu ,\lq\lq{}Improvement And Experimental Evaluation Bellman-Ford Algorithm\rq\rq{}, \textit{International Conference on Advanced Information and Communication Technology for Education (ICAICTE 2013)}

\end{document}